\section{General Approach: A Two Step Optimisation}
\label{ProblemFormulation}

In WSNs, sensors often suffer from unreliable radio environment that is noisy and error prone which results in higher energy drain rate. 
%Sensors are typically densely deployed with minimal maintenance requirements to track and monitor events which the data are reported to a centralised data collection node for analysis. 
%%WSNs require reliable event detection and communication to ensure important sensed data to be transmitted and received as intended. The network also has to be fully functional for a long period as the sensors are easily deployed in difficult to reach areas where only minimal maintenance is possible. 
WSNs require reliable communication to ensure important data to be transmitted and received as intended.
The network also has to be fully functional for the maximum period possible especially in the case where only certain nodes could reach the sink node.
%%Sensors could easily be deployed in difficult to reach areas where only minimal maintenance is possible. 
%typically densely deployed with minimal maintenance requirements.
% which interference is the main concern for a reliable communication. 
In order to fulfil these requirements, a two step optimisation approaches were used where the main goals are to;
%to overcome the interference and energy problems where the goals are to; 
(i) reduce interference through multichannel and (ii) maximise the network lifetime by reconfiguring the topology. These approaches are briefly described in the next section.

%There are two steps main goals in this study which are done is steps; (i) to maximise the throughput through multichannel and (ii) to maximise the network lifetime. The approaches to achieve these goals are briefly described.

%stable communications except in multimedia WSNs where it requires tight network connectivity requirement. 

%automate tracking and monitoring events that reports to a centralised data collection node (sink).

%which results in higher energy drain rate. This rise the problems (concern) to (fulfil both criteria) Sensors are typically deployed to automate track and monitor events in as they 
%This () sensors to require minimal maintenance and to maintain working without interruption for a longer period.
%There are two main goals in this study which are done is steps; (i) to maximise the throughput and (ii) to maximise the network lifetime. The approaches to achieve these goals are briefly described.

%\subsection{Maximise Throughput}
\subsection{Reduce Interference}
In single channel MAC protocols, nodes are configured to use a single channel throughout the nodes lifetimes. Multichannel has the advantage of an increase in robustness against external and intra nodes interference which as a result, improves the network traffic flow which reduces packet loss and maximise the overall throughput.
Multichannel is a preferable solution to improve resilience against interference and maintain reliable communications. However, not all channels are free from interference; thus, there is a gain to hop to another channel when the quality of the channel deteriorates. 
The authors in \cite{homearea} found that the channel reliability changes over time in non cyclic manner, thus no specific channels could achieve a long term reliability. 

Multichannel Cross-Layer Routing Protocol (MCRP) \cite{mcrp} is introduced with extension results from the previously published paper where the protocol assigns and thoroughly checks for channels that are free or have low interference for nodes channel allocation. All available channels are considered and the channels reliability are checked during run time for precision to ensure infrequent channel hopping processes to be invoked which could have an effect on the network connectivity. MCRP is briefly explained in Section \ref{MCRP} for completeness to relate to the further improvement done to the topology.
%is require to ensure network connectivity.

%Frequency agile MAC protocols on the other hand, allow the nodes to switch to different channels during run time. This is possible as recent radio chips take less than 100μs to switch to a different channel. The channel switching delay is negligible in the WSN context where the packet rates are low. This makes multichannel attractive for use in WSNs. 

\subsection{Maximise Lifetime}
%%%need??? In WSNs, it is necessary to estimate the nodes' power consumption before they are deployed to enable accurate forecast of the energy consumption. The estimations are used to determine the nodes lifetime before maintenance and batteries replacements are required in order to have a functional network. Unfortunately, the node lifetime is very dependent on the radio environment that can be unstable, noisy and error prone which makes energy consumption to vary \cite{alexlifetime}. 
The network lifetime depends on various factors such as the network architecture and protocols, channel characteristics, energy consumption model and the network lifetime definition. In order to increase the network lifetime, these information regarding the channel and residual energy of the sensors should be exploited.
%///relating MCRP to finding optimal tree to consider the residual energy
Multichannel protocol not only could reduce the end to end delay, it also helps to ensure minimal packet retransmissions thus consume less energy during communications as the effect from multichannel. However, it is not for certain that the topology has the energy optimal routes as the channels would have different effect on the nodes. While the node uses a better channel than previously, another path from the node on the new channel might gives a better result. 
Multichannel helps to maximise the throughput but it does not maximise the network lifetime.
%%depending on multichannel is not enough 
MCRP consumes less energy than in other cases as the effect from multichannel. 

In order to increase the energy efficiency thus network lifetime, MCRP needs to reconstruct the topology based on the available energy of the nodes and the link conditions gradually to avoid breaking any current connectivity. The energy-based tree reconfiguration is describes in Section \ref{OptimalTree} and the results show prolonged lifetime by 6.2 times more when the optimal tree is found in the 500 nodes network.