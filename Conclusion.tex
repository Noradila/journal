\section{Conclusion}
\label{Conclusion}

In this paper, two steps optimisation approaches were presented that reduce the effect of interference by implementing MCRP, a multichannel cross-layer routing protocol, and maximise the network lifetime by reconfiguring the topology to find the optimal tree. 
MCRP shows high packet reception rate of nearly 100\% in simulations and 80\%-90\% in hardware results. In addition, MCRP reduces the energy consumption by an average of 3 times during communications as the effect of multichannel.
%in addition to the reduced energy consumptions during communications as the effect of multichannel.  
The energy-based tree reconfiguration is proposed to further improve the multichannel network by considering the energy level of each sensor. 
%The sensors are balanced in term of the ability to route packets based on the residual energy available.
%The sensors residual energy are balanced
%The equation and algorithm used in finding the optimal tree are explained. 
It is aimed to enable the network to be fully functional for a longer period of time by maximising the minimum sensor energy level and enable the sensors to have similar lifetime.
%through topology reconstruction. 
The results showed an increase in the network lifetime by 8.3 times more for the optimal tree compared to the initial tree of 0.3\% in a 500 nodes system.